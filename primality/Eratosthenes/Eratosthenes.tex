\documentclass[10pt,a4paper]{article}
\title{Sieve of Eratosthenes}
\date{}

\usepackage[utf8]{inputenc}
\usepackage{amsmath}
\usepackage{amsfonts}
\usepackage{amssymb}
% For algorithms
\usepackage[linesnumbered,ruled,lined,boxed,commentsnumbered]{algorithm2e}

\usepackage{titling}

\setlength{\droptitle}{-10em}

\begin{document}
\maketitle

\begin{algorithm}[H]
	\KwData{A postive integer $n \geq 2$}
	\KwResult{An array $M[2 \text{ .. } n]$ such that $M[i] = 0$ if $i$ is
	prime and 1 if $i$ is composite}
	\BlankLine
	\Begin {
		Let $M[2 \text{ .. } n]$ be initialized with all elements = 0\;
		\BlankLine
		\For{$i \leftarrow 2$ to $n$} {
			\If{$M[i] \neq 1$} {
				\For{$j \leftarrow 2i$ and $j \leq n$} {
					$M[j] \leftarrow 1$\;
					$j \leftarrow j+i$\;
				}
			}
			$i \leftarrow i+1$\;
		}
		return $M$\;
	}
\caption{Sieve of Eratosthenes}
\end{algorithm}

\section{Proof of Correctness (By Induction)}
	\subsection{Inductive Hypothesis}
	Let $P(i) :=$ At the beginning of the outer for loop, all prime	numbers
	$< i$ have been found (set to 0 in M) and the multiples of all these
	primes have been marked (set to 1) in M.       

	\subsection{Base Case}
	Initially, $i = 2$. As there are no prime numbers less than 2 and our
	indexing for M begins from 2, P(2) is trivially true.

	\subsection{Inductive Step}
	Assume the $P(k)$ to be true for some $k \geq 2$. When $i = k+1$, we have
	two cases. If $k$ is composite, it can expressed as the product of
	primes i.e $k = p_1^{\alpha_{1}} p_2^{\alpha_{2}} \text{ ... } p_n^{\alpha_{n}}$
	where $p_{j}$ is a prime number $< k$. According to our assumption, this would mean
	that $k$ has already been marked (set to 1 in M). The inner loop will not execute.
	\BlankLine
	If $k$ is indeed prime, then none of the previous marking procedures would
	have affected $M[k]$. As $M[k]$ was initially set 0, it will continue to remain 0.
	In addition to this, the inner loop will execute successfully and mark all multiples
	of $k$.	
	\BlankLine
	In both the cases, we have now found all prime numbers less than $k+1$ and marked
	their multiples i.e $P(k+1)$ holds true.

	\subsection{Termination}
	The procedure terminates when $i > n$. As $i$ is always incremented by 1, $i$ will always
	be equal to $n+1$ when the procedure terminates. $\therefore P(n+1)$ is true i.e all prime
	numbers $ < n+1$ have been found. $\blacksquare$
\end{document}
