\documentclass[10pt,a4paper]{article}
\title{Bubble Sort}
\date{}

\usepackage[utf8]{inputenc}
\usepackage{amsmath}
\usepackage{amsfonts}
\usepackage{amssymb}
% For algorithms
\usepackage[linesnumbered,ruled,lined,boxed,commentsnumbered]{algorithm2e}

\usepackage{titling}

\setlength{\droptitle}{-10em}

\begin{document}
\maketitle

\begin{algorithm}[H]
 \KwData{An array A[0 ... n-1] of size n}
 \KwResult{Sorted array A in non decreasing order i.e $A[0]<=A[1]<=...<=A[n-1]$ }
  \BlankLine
  \Begin {
  \For{$i \leftarrow 0$ to $n-2$} {
	\tcc{Shift the maximum element of sub-array A[0..n-i-1] to A[n-i-1] by adjacent pairwise swapping}
  	\For{$j \leftarrow 0$ to $n-i-2$} {
  		\If{$A[j] > A[j+1]$}{
			\BlankLine
			\tcc{Swap A[j] and A[j+1]}
			$tmp \leftarrow A[j]$\;
			$A[j] \leftarrow A[j+1]$\;
			$A[j+1] \leftarrow tmp$\;
  		}
  		$j \leftarrow j+1$\;
  	}
  	\BlankLine
  	$i \leftarrow i+1$\;
 }
 }
 \caption{Bubble Sort}
\end{algorithm}

\section{Proof of Correctness}
\subsection{Invariant}
At the beginning of each outer for loop, the sub array $A[n-i \text{ .. } n-1]$ consists of the first $(n-1)-(n-i)+1 = i$ largest elements of the entire array A, in sorted order i.e $A[n-i] <= A[n-i+1] <= \text{ .. } <= A[n-1]$.

\subsection{Initialization}
Initially, $i = 0$ and hence the sub array $A[n \text{ .. } n-1]$ is an empty list and consists of 0 
elements. As the list is empty, the invariant trivially holds.

\subsection{Maintenance}
The inner for loop runs from $j \leftarrow 0$ through $n-i-2$. By performing adjacent comparisons and swaps between $A[j]$ and $A[j+1]$, we ensure that the the maximum element of sub array $A[0 \text{ .. } n-i-1]$ is shifted and placed at $A[n-i-1]$ giving us the first $(n-1) - (n-i-1) + 1 = i+1$ largest elements in $A[n-i-1 \text{ .. } n-1]$, that too in sorted order. Incrementing $i$ to $i+1$ then makes the invariant hold at the start of the next iteration.

\subsection{Termination}
The procedure terminates when $ i > n-2$. As $i$ is always incremented by 1, when the
outer while loop terminates $i$ will always be equal to $n-1$. According to the invariant, 
when $i = n-1$, the sub array $A[1 \text{ .. } n-1]$ must contain the first $(n-1)$ largest elements
of A in sorted order. By elimination, $A[0]$ must now contain the $n^{th}$ largest element (the smallest element).
In other words, the array $A$ is now sorted. $\blacksquare$

\end{document}
