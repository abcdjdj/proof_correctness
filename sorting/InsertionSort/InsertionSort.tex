\documentclass[10pt,a4paper]{article}
\title{Insertion Sort}
\date{}

\usepackage[utf8]{inputenc}
\usepackage{amsmath}
\usepackage{amsfonts}
\usepackage{amssymb}
% For algorithms
\usepackage[]{algorithm2e}

\usepackage{titling}

\setlength{\droptitle}{-10em} 

\begin{document}
\maketitle

\begin{algorithm}[H]
 \KwData{An array A[0 ... n-1] of size n}
 \KwResult{Sorted array A in non decreasing order i.e $A[0]<=A[1]<=...<=A[n-1]$ }
 $i \leftarrow 1$\; 
 \While{$i < n$} {
  $tmp \leftarrow A[i]$\;
  $j \leftarrow i-1$\;
  \tcc{Insert A[i] into sorted sub-array A[0..i-1]}
  	\While{$j >= 0$ and $A[j] > tmp$} {
  		$A[j+1] \leftarrow A[j]$\;
  		$j \leftarrow j - 1$\;
  	}
  	$A[j+1] \leftarrow tmp$\;
  	$i \leftarrow i + 1$\;
 }
 \caption{Insertion Sort}
\end{algorithm}

\section{Proof of Correctness}
\subsection{Invariant}
At the beginning of each outer loop, the sub array $A[0 \text{ .. } i-1]$ is sorted 
and consists of the same elements that were originally in $A[0 \text{ .. } i-1]$.

\subsection{Initialization}
Initially, $i = 1$ and hence the sub array $A[0 \text{ .. } i-1]$ consists of only one 
element i.e $A[0]$. As a single element is trivially sorted, the invariant holds.

\subsection{Maintenance}
The inner while loop shifts elements $A[i-1]$, $A[i-2]$ and so on by one position to
the right until the right position for $A[i]$ is found. The element $A[i]$ is then 
inserted into it's correct position. The sub array $A[0 \text{ .. } i = j+1]$ contains 
the original elements of $A[0 \text{ .. } i = j+1]$  but in ascending order. Incrementing $i$
to $i + 1$ for the next iteration, then preserves the invariant.

\subsection{Termination}
The procedure terminates when $ i >= n$. As $i$ is always incremented by 1, when the
outer while loop terminates $i$ will always be equal to $n$. According to the invariant, when $i = n$,
the sub array $A[0 \text{ .. } n-1]$ must contain the same original elements of the array
but in ascending order. In other words, the array $A$ is now sorted. $\blacksquare$

	
\end{document}